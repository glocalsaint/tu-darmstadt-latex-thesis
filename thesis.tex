% Include header.tex with configuration

% document class
\documentclass[
  accentcolor=tud9c,
  bibliography=totoc,
  listof=totoc,
  oneside,
  numbersubsubsec,
  fontsize=12pt,
  type=sta,
  colorback
]{tudthesis}

% to-do notes
\usepackage[english]{todonotes}

% no indentation for new paragraphs
\setlength{\parindent}{0pt}

% input encoding
\usepackage[utf8]{inputenc}

% font encoding
\usepackage[T1]{fontenc}

% german or english document
% \usepackage[ngerman]{babel}
\usepackage[english]{babel}

% set line spacing to 1.5
\usepackage{setspace}
\onehalfspacing

% multiple rows inside a table
\usepackage{multirow}

% german quotes and package important for biblatex
\usepackage[german=quotes]{csquotes}

% consistent text for graphics
\usepackage{psfrag}

% bibliography with biblatex
\usepackage[style=numeric, firstinits=true, backend=bibtex]{biblatex}
% same style for URLs
\urlstyle{same}
% title not italic
\DeclareFieldFormat{title}{#1\isdot}
% colon between author and title
\renewcommand{\labelnamepunct}{\addcolon\space}
% use semicolon when multiple authors
\renewcommand{\multinamedelim}{\addsemicolon\space}
% seperate last author with semicolon
\renewcommand{\finalnamedelim}{\addsemicolon\space}
% format name
\DeclareNameFormat{default}{\usebibmacro{name:last-first}{#1}{#4}{#5}{#7}\usebibmacro{name:andothers}}
% text for url
\DefineBibliographyStrings{ngerman}{urlseen={abgerufen am}}

% biblatex
\bibliography{bibliography/bibliography}

% draw beautiful graphs and import MATLAB files
\usepackage{tikz}
\usetikzlibrary{shapes,arrows}
\usepackage{pgfplots}
\pgfplotsset{compat=newest}

% electronic circuits
\usetikzlibrary{circuits.ee.IEC}

% fill bar charts with patterns
\usetikzlibrary{patterns}

% Euro symbol with symbol behind value
\usepackage[right]{eurosym}

% properly underlined url in bibliography
\usepackage{url}

% adds align command
\usepackage{amsmath}

% use graphics
\usepackage{graphicx}

% display pics next to each other, don't use older subfigure package
\usepackage{subfig}

% use nicer tables with footnotes
\usepackage{ctable}

% force all floats to render with \FloatBarrier
\usepackage{placeins}

% acronyms and nomenclature
\usepackage[acronym, toc, nonumberlist]{glossaries}

% links within latex
\usepackage[bookmarks]{hyperref}
\hypersetup{
  linktocpage=false,
  breaklinks=true,
  bookmarksnumbered=true,
  colorlinks=false,
  pdftitle=Technical and economical analysis of an electric vehicle including design of a charging station fed by renewable energy,
  pdfauthor=Mirco Zeiss,
}

% break long URLs in toc
\usepackage{breakurl}
\usepackage{pgfplots}
\usetikzlibrary{pgfplots.statistics}
% Use colors for corporate design
\input{features/custom-colors}

\newglossary[sylg]{symbolslist}{syi}{syo}{Nomenclature}
\makeglossaries

\input{features/acronyms}

\begin{document}

% Titelseite der Studienarbeit

% #1: Titel der Arbeit in Erstsprache (z.B. Deutsch)
% #2: Titel der Arbeit in der zweiten Sprache (z.B. Englisch)
% \thesistitle{Technische und wirtschaftliche Analyse eines Elektrofahrzeugs einschließlich Auslegung einer regenerativen Stromtankstelle}{}
\thesistitle{Viswanath and economical analysis of an electric vehicle including design of a charging station fed by renewable energy}{}

% Name des Autors
\author{Mirco Zeiss}

% Fachbereich an dem die Arbeit durchgeführt wurde
% \department{Fachbereich Elektro- und \\
%       Informationstechnik}
\department{Department of Electrical Engineering and Information Technology}

% Institut an dem die Arbeit durchgeführt wurde
\group{Electrical Energy Conversion}

% Individuelles Datum
% \date{21.03.2011}
\date{03/21/2011}

% #1: Name von Gutachter 1
% #2: Name von Gutachter 2
% #3: Name von Gutachter 3 (optional)
\referee{Prof. \#1}{Prof. \#2}{}
\affidavit{V. Vadhri}
\makethesistitle

\abstract{Viswanath Vadhri is a good boy}

% Start with roman page numbering
\pagenumbering{Roman}

% Comment out the following two lines before printing your final document
\listoftodos
\newpage

% Print table of contents
\tableofcontents

\newpage

% Print list of figures
\listoffigures

\newpage

% Print list of tables
\listoftables

\newpage

% Define custom style for glossaries
\newglossarystyle{mystyle}{%
  % full stop after every description
	\renewcommand*{\glspostdescription}{}%
	% put the glossary in a longtable environment:
  % left alignment, no white in front und three columns
	\renewenvironment{theglossary}{\begin{longtable}[l]{@{}lcl}}{\end{longtable}}
	% have nothing after \begin{theglossary}:
	\renewcommand*{\glossaryheader}{}%
  % uncomment following line if you want headings
	% \renewcommand*{\glossaryheader}{\bfseries Symbol & \bfseries Unit & \bfseries Description \endhead}%
	% have nothing between glossary groups (next two commands):
	\renewcommand*{\glsgroupheading}[1]{}%
	% Suppress the vertical gap at the start of each group
	\renewcommand*{\glsgroupskip}{}%
	% set how each entry should appear:
	\renewcommand*{\glossaryentryfield}[5]{%
		\glstarget{##1}{##2}	% Name
		& ##4					        % Symbol
		& ##3					        % Description
		% & ##5					      % Page list
		\\% end of row
	}%
	% Sub entries treated the same as level 0 entries:
	\renewcommand*{\glossarysubentryfield}[6]{%
		\glossaryentryfield{##2}{##4}{##3}
	}%
}

% Print acronyms
\printglossary[type=acronym, title=Acronyms, toctitle=Acronyms, style=mystyle]

% Print list of symbols
\printglossary[type=symbolslist, style=mystyle]
\newpage

% Save current value of roman page number
\newcounter{romanpagenumbers}
\setcounter{romanpagenumbers}{\value{page}}

% Change to arabic page numbers
\pagenumbering{arabic}

% --- Equations ---
\chapter{Equations}

Equation for extraterrestrial radiation with \gls{I0}, \gls{SC}, \gls{n}, \gls{L}, \gls{delta} and \gls{HSR}.


Equation for average wind power.

\begin{equation}
	\overline P_w = \frac{c_1}{T} \int_0^T v_w^3\,dt \ne c_1 \left(\frac{1}{T}\int_0^T v_w\,dt\right)^3 = c_1 \cdot \overline{v}_w^3.
\end{equation}

% --- Acronyms ---
\chapter{Acronyms}

Use package \href{ftp://ftp.dante.de/tex-archive/macros/latex/contrib/glossaries/glossariesbegin.pdf}{glossaries}.

 Viswanath Vadhri Lorem ipsum dolor sit amet, consectetuer adipiscing elit \textbf{\gls{CO2} for the first time}, sed diam nonummy nibh euismod tincidunt ut laoreet dolore magna aliquam erat volutpat. Ut wisi enim ad minim veniam, quis nostrud exerci tation ullamcorper suscipit lobortis nisl ut aliquip ex ea commodo consequat. Duis autem vel eum iriure dolor in hendrerit in vulputate velit esse molestie consequat \textbf{\gls{CO2} for the second time}, vel illum dolore eu feugiat nulla facilisis at vero eros et accumsan et iusto odio dignissim qui blandit praesent luptatum zzril delenit augue duis dolore te feugait nulla facilisi. Nam liber tempor cum soluta nobis eleifend option congue nihil imperdiet doming id quod mazim placerat facer possim assum. Typi non habent claritatem insitam; est usus legentis in iis qui facit eorum claritatem. 

Lorem ipsum dolor sit amet, consectetuer adipiscing elit \textbf{\gls{NEDC} for the first time}, sed diam nonummy nibh euismod tincidunt ut laoreet dolore magna aliquam erat volutpat. Ut wisi enim ad minim veniam, quis nostrud exerci tation ullamcorper suscipit lobortis nisl ut aliquip ex ea commodo consequat. Duis autem vel eum iriure dolor in hendrerit in vulputate velit esse molestie consequat \textbf{\gls{NEDC} for the second time}, vel illum dolore eu feugiat nulla facilisis at vero eros et accumsan et iusto odio dignissim qui blandit praesent luptatum zzril delenit augue duis dolore te feugait nulla facilisi. Nam liber tempor cum soluta nobis eleifend option congue nihil imperdiet doming id quod mazim placerat facer possim assum. Typi non habent claritatem insitam; est usus legentis in iis qui facit eorum claritatem. 

Lorem ipsum dolor sit amet, consectetuer adipiscing elit \textbf{\gls{D} for the first time}, sed diam nonummy nibh euismod tincidunt ut laoreet dolore magna aliquam erat volutpat. Ut wisi enim ad minim veniam, quis nostrud exerci tation ullamcorper suscipit lobortis nisl ut aliquip ex ea commodo consequat. Duis autem vel eum iriure dolor in hendrerit in vulputate velit esse molestie consequat \textbf{\gls{D} for the second time}, vel illum dolore eu feugiat nulla facilisis at vero eros et accumsan et iusto odio dignissim qui blandit praesent luptatum zzril delenit augue duis dolore te feugait nulla facilisi. Nam liber tempor cum soluta nobis eleifend option congue nihil imperdiet doming id quod mazim placerat facer possim assum. Typi non habent claritatem insitam; est usus legentis in iis qui facit eorum claritatem. 

% --- Simple citations ---
\chapter{Citations}

This could be a sentence from a book.\cite{wagner2010}

% --- Page numbering ---
\chapter{Page numbering}

\input{chapters/page-numbering}

% --- Complex tables ---
\chapter{Table}

\input{chapters/table}

% --- To-do notes ---
\chapter{To-do notes}

\input{chapters/todonotes}

% --- Images side by side ---
\chapter{Images}

\input{chapters/images}

% --- Bar chart ---
\chapter{Bar charts}

This chapter shows how to render a simple bar chart, a stacked bar chart and a grouped bar chart.

% --- Simple bar chart ---
\section{Simple bar chart}

\input{chapters/bar-chart}

\newpage


\begin{equation}
  \glsentrytext{I0} = \frac{24}{\pi} \glsentrytext{SC} \left[1+0,034\cos\left(\frac{360\glsentrytext{n}}{365}\right)\right](\cos \glsentrytext{L} \cos \glsentrytext{delta} \sin \glsentrytext{HSR} + \glsentrytext{HSR} \sin \glsentrytext{L} \sin \glsentrytext{delta})
\end{equation}

\begin{tikzpicture}
\begin{axis}[
  boxplot/draw direction=y,
  boxplot/variable width,
  boxplot/every box/.style={fill=blau_2b},
  width= 0.9\textwidth,
  height = 0.6\textwidth,
  ymajorgrids = true,
  xticklabels={Index 0, Index 1, Index 2, Index 3, Index 4},
  xtick={1,2,3,4,5}
]
    \addplot+[boxplot prepared={
      median=15,
      upper quartile=17,
      lower quartile=10,
      upper whisker=20,
      lower whisker=5
    },
    ] coordinates {};
      \addplot+[boxplot prepared={
      median=15,
      upper quartile=17,
      lower quartile=10,
      upper whisker=20,
      lower whisker=5
    },
    ] coordinates {};
      \addplot+[boxplot prepared={
      median=15,
      upper quartile=17,
      lower quartile=10,
      upper whisker=20,
      lower whisker=5
    },
    ] coordinates {};
      \addplot+[boxplot prepared={
      median=15,
      upper quartile=17,
      lower quartile=10,
      upper whisker=20,
      lower whisker=5
    },
    ] coordinates {};
  \addplot+[boxplot prepared={
      median=15,
      upper quartile=17,
      lower quartile=10,
      upper whisker=20,
      lower whisker=5
    },
    ] coordinates {};

\end{axis}
\end{tikzpicture}


\begin{tikzpicture}
  \begin{axis}
    [
    boxplot/draw direction=y,
  boxplot/variable width,
  boxplot/every box/.style={fill=blau_2b},
  width= 0.9\textwidth,
  height = 0.6\textwidth,
  ymajorgrids = true,
    ytick={1,2,3},
    yticklabels={Index 0, Index 1, Index 2},
    ]
    \addplot+[
    boxplot prepared={
      median=1,
      upper quartile=1.2,
      lower quartile=0.4,
      upper whisker=1.5,
      lower whisker=0.2
    },
    ] coordinates {};
    \addplot+[
    boxplot prepared={
      median=2,
      upper quartile=2.3,
      lower quartile=1.5,
      upper whisker=2.7,
      lower whisker=1
    },
    ] coordinates {};
    \addplot+[
    boxplot prepared={
      median=0.7,
      upper quartile=1.4,
      lower quartile=0.5,
      upper whisker=1.9,
      lower whisker=0.1
    },
    ] coordinates {};
  \end{axis}
\end{tikzpicture}
% --- Stacked bar chart ---
\section{Stacked bar chart}

\input{chapters/stacked-bar-chart}

\newpage

% --- Grouped bar chart ---
\section{Grouped bar chart}

\input{chapters/grouped-bar-chart}

% --- Pie chart ---
\chapter{Pie chart}

\input{chapters/piechart}

% --- Line graph ---
\chapter{Line graph}

\input{chapters/line-graph}

% --- Line graph ---
\chapter{Two y-axes}

\input{chapters/two-y-axes}

% --- Text replacement ---
\chapter{Text replacement}

\input{chapters/text-replacement}

% --- Electronic circuits ---
\chapter{Electronic circuits}

Use the \textbf{tikz} package to draw electronic circuits and more.

\input{chapters/circuits}

\newpage

% Continue with roman page numbering
\pagenumbering{Roman}
\setcounter{page}{\value{romanpagenumbers}}

% Print the bibliography
\printbibliography
  
% End of document
\end{document}